\section{Desiderata for Columbus}
Based on the limitations of current approaches and the motivating examples listed above, we have 
identified the following desiderata for Columbus:

\begin{itemize}
  \item Collaborative: Users should be able to share data and code in an easy and intuitive manner.
  \item Feature rich: A data cleaning and integration platform is only as good as the
    functionality it provides. Columbus should have the capability to perform all operations
    present in common data cleaning workflows.
  \item Easy to install: Lay users should be able to install and use packages with a few
    clicks. Also, the system itself should be easy to deploy and use. A small team of 4-5 data 
    scientists should be able to easily set up a Columbus deployment for their internal use. 
  \item Extensible: If a package to perform a particular task does not exist on Columbus, a user 
    should be able to create a new package for it. Other users should then be able to install and
    use this package.
  \item Scalable: A team using Columbus should be able to scale up or scale down their deployment 
    depending on the number of users they want to support. Also, packages created for Columbus 
    should be able to leverage cloud services such as distributed computing, serverless functions,
    cloud storage etc. in order to scale to larger datasets. Package developers should not need to
    separately set up cloud infrastructure for creating a package, they should be able to leverage
    this infrastructure through Columbus in an easy and intuitive manner.
  \item Composable: Users should be able to compose packages together to create complex workflows
    suited to their needs.
  \item Development environment agnostic: Package developers should be free to develop packages in
    the programming language of their choice. It should also be possible to intuitively compose 
    together two or more packages developed in different programming languages.
  \item Secure: Columbus should handle user authorization for data access. It should also provide
    authentication and authorization as a service to third parties so that package developers do
    not need to rewrite authentication logic for each package.
\end{itemize}
